% ---
% Este capítulo, apresenta os conceitos sobre o algoritmo genético
% ---

\chapter{Implementação do algoritmo genético}
\label{chap:implementGA}

Para implementação do algoritmo foi utilizado a linguagem de programação JAVA de forma a aproveitar o paradigma de orientação a objetos e reaproveitamento de código. A estrutura do programa permite com classes básicas formar a lógica do algoritmo e com classes derivadas especializar o algoritmo para especificações de cada problema. Dessa forma com pouco código extra sendo feito é possível reaproveitar as classes já definidas, usando o polimorfismo inerente a linguagens OO, é possível reescrever e acrescentar as partes necessárias para embutir conhecimento específico no problema.

\section{Modelo de Ising}

--- Inserir uma explicação simples sobre o modelo de Ising

O algoritmo genético simples será usado para apresentar uma solução para modelo de Ising em uma matriz \(10 \times 10\), com cada nó (spin) representado pelos possíveis valores $-1$ e $1$. 

Sendo \(\Lambda = \left[-5,5\right]^2\) as posições da matriz, e \(\sigma_t \in \{-1,1\}\) representando os valores de cada spin em determinada posição da matriz, e seja \( (\sigma_t)_{t \in \Lambda} = \sigma \) um conjunto de valores da matriz. 

A função de avaliação que deve ser minimizada é dada por \( H(\sigma) = -\sum \limits_{<t,t'>}\sigma_t \sigma_{t'} \) onde $t$ e $t'$ representam dois spins vizinhos.   

\section{Codificação}

A classe cromossomo será usada para representar cada solução para o problema e conterá um vetor de valores inteiros de tamanho $100$, representando assim a matriz \( 10 \times 10 \), alinhando uma linha da matriz na sequencia da outra no vetor. Assim o nosso genótipo é uma cadeia de inteiros, onde o fenótipo será a matriz correspondente quebrando o vetor em linhas de tamanho $10$.

Exemplo da relação genótipo x fenótipo para uma matriz \( 3 \times 3 \):
\begin{align*}
\text{cromossomo} &= \left[-1, 1, 1, 1, -1, 1, 1, 1, -1 \right]\\
\text{fenótipo} &= \begin{bmatrix}
-1 & 1 & 1 \\
1 & -1 & 1 \\
1 & 1 & -1
\end{bmatrix}
\end{align*}


\section{Seleção e avaliação}

\section{Crossover e mutação}

\section{Resultados}