% ---
% Este capítulo, apresenta os conceitos sobre o algoritmo genético
% ---

\chapter{Algoritmos Genéticos}
\label{chap:GA}

O algoritmo genético (GA) parte de uma população de \textit{cromossomos} que representam as soluções candidatas para o problema de otimização. Cada uma das soluções é avaliada de acordo com critérios inerentes do problema, para posteriormente serem selecionados e combinados de forma a criar novas soluções candidatas.

Já é possível perceber uma das vantagens do algoritmo, que ele irá fazer uma avaliação direta das soluções e de forma paralela. Por exemplo em problemas NP-Completos onde são difíceis de se obter soluções numéricas eficientes, mas é possível testar soluções em tempo polinomial, o algoritmo se torna atrativos com essas características.

\section{Seleção}

\section{Crossover}

\section{Mutação}

\section{Esquemas}

\section{Fundamentos teóricos}