%% abtex2-modelo-trabalho-academico.tex, v-1.9.7 laurocesar
%% Copyright 2012-2018 by abnTeX2 group at http://www.abntex.net.br/ 
%%
%% This work may be distributed and/or modified under the
%% conditions of the LaTeX Project Public License, either version 1.3
%% of this license or (at your option) any later version.
%% The latest version of this license is in
%%   http://www.latex-project.org/lppl.txt
%% and version 1.3 or later is part of all distributions of LaTeX
%% version 2005/12/01 or later.
%%
%% This work has the LPPL maintenance status `maintained'.
%% 
%% The Current Maintainer of this work is the abnTeX2 team, led
%% by Lauro César Araujo. Further information are available on 
%% http://www.abntex.net.br/
%%
%% This work consists of the files abntex2-modelo-trabalho-academico.tex,
%% abntex2-modelo-include-comandos and abntex2-modelo-references.bib
%%

% ------------------------------------------------------------------------
% ------------------------------------------------------------------------
% abnTeX2: Modelo de Trabalho Academico (tese de doutorado, dissertacao de
% mestrado e trabalhos monograficos em geral) em conformidade com 
% ABNT NBR 14724:2011: Informacao e documentacao - Trabalhos academicos -
% Apresentacao
% ------------------------------------------------------------------------
% ------------------------------------------------------------------------

\documentclass[
	% -- opções da classe memoir --
	12pt,				% tamanho da fonte
	openright,			% capítulos começam em pág ímpar (insere página vazia caso preciso)
	twoside,			% para impressão em recto e verso. Oposto a oneside
	a4paper,			% tamanho do papel. 
	% -- opções da classe abntex2 --
	%chapter=TITLE,		% títulos de capítulos convertidos em letras maiúsculas
	%section=TITLE,		% títulos de seções convertidos em letras maiúsculas
	%subsection=TITLE,	% títulos de subseções convertidos em letras maiúsculas
	%subsubsection=TITLE,% títulos de subsubseções convertidos em letras maiúsculas
	% -- opções do pacote babel --
	english,			% idioma adicional para hifenização
	%french,				% idioma adicional para hifenização
	%spanish,			% idioma adicional para hifenização
	brazil				% o último idioma é o principal do documento
	]{abntex2}

% ---
% Pacotes básicos 
% ---
\usepackage{lmodern}			% Usa a fonte Latin Modern			
\usepackage[T1]{fontenc}		% Selecao de codigos de fonte.
\usepackage[utf8]{inputenc}		% Codificacao do documento (conversão automática dos acentos)
\usepackage{indentfirst}		% Indenta o primeiro parágrafo de cada seção.
\usepackage{color}				% Controle das cores
\usepackage{graphicx}			% Inclusão de gráficos
\usepackage{microtype} 			% para melhorias de justificação
% ---
		
% ---
% Pacotes adicionais, usados apenas no âmbito do Modelo Canônico do abnteX2
% ---
\usepackage{lipsum}				% para geração de dummy text
% ---

% ---
% Pacotes de citações
% ---
\usepackage[brazilian,hyperpageref]{backref}	 % Paginas com as citações na bibl
\usepackage[alf, abnt-emphasize=bf]{abntex2cite}	% Citações padrão ABNT

% --- 
% CONFIGURAÇÕES DE PACOTES
% --- 

% ---
% Configurações do pacote backref
% Usado sem a opção hyperpageref de backref
%\renewcommand{\backrefpagesname}{Citado na(s) página(s):~}
\renewcommand{\backrefpagesname}{}
% Texto padrão antes do número das páginas
\renewcommand{\backref}{}
% Define os textos da citação
%\renewcommand*{\backrefalt}[4]{
%	\ifcase #1 %
%		Nenhuma citação no texto.%
%	\or
%		Citado na página #2.%
%	\else
%		Citado #1 vezes nas páginas #2.%
%	\fi}%
\renewcommand*{\backrefalt}[4]{}%
% workaround referências a anexos
\newcommand{\refanexo}[1]{\hyperref[#1]{Anexo~\ref{#1}}}
% --- Pacote próprios
\usepackage{amsmath}
\usepackage{amsfonts}
\usepackage{amssymb}
\usepackage{bm}
\usepackage{bbm}
\usepackage{mathrsfs}
%\usepackage{graphicx}
%\usepackage{float}
\usepackage{url}
\usepackage{verbatim}
\usepackage{babel}
\usepackage{subcaption}
\usepackage[portuguesekw, boxed]{algorithm2e}
\usepackage[font=small,labelfont=bf]{caption}
% --- 
% --- Configurações extras
% Acerta nomes dos algoritmos
\SetAlgorithmName{Algoritmo}{Algoritmo}{Lista de algoritmos}
% ---
% Informações de dados para CAPA e FOLHA DE ROSTO
% ---
\titulo{Algoritmos Genéticos\\Um algoritmo evolucionário para otimização}
\autor{Fabricio Kassardjian}
\local{Brasil}
\data{\today}
\orientador{Anatoli Iambartsev}
\instituicao{%
  Universidade de São Paulo -- USP
  \par
  Insituto de Matemática e Estatística
  \par
  Bacharelado em Matemática Aplicada e Computacional}
\tipotrabalho{Monografia}
% O preambulo deve conter o tipo do trabalho, o objetivo, 
% o nome da instituição e a área de concentração 
\preambulo{Monografia de final de curso, Instuto de Matemática e Estatística - USP, Matemática Aplicada}
% ---


% ---
% Configurações de aparência do PDF final

% alterando o aspecto da cor azul
\definecolor{blue}{RGB}{41,5,195}

% informações do PDF
\makeatletter
\hypersetup{
     	%pagebackref=true,
		pdftitle={\@title}, 
		pdfauthor={\@author},
    	pdfsubject={\imprimirpreambulo},
	    pdfcreator={LaTeX with abnTeX2},
		pdfkeywords={abnt}{latex}{abntex}{abntex2}{trabalho acadêmico}, 
		colorlinks=true,       		% false: boxed links; true: colored links
    	linkcolor=blue,          	% color of internal links
    	citecolor=blue,        		% color of links to bibliography
    	filecolor=magenta,      		% color of file links
		urlcolor=blue,
		bookmarksdepth=4
}
\makeatother
% --- 

% ---
% Posiciona figuras e tabelas no topo da página quando adicionadas sozinhas
% em um página em branco. Ver https://github.com/abntex/abntex2/issues/170
\makeatletter
\setlength{\@fptop}{5pt} % Set distance from top of page to first float
\makeatother
% ---

% ---
% Possibilita criação de Quadros e Lista de quadros.
% Ver https://github.com/abntex/abntex2/issues/176
%
\newcommand{\quadroname}{Quadro}
\newcommand{\listofquadrosname}{Lista de quadros}

\newfloat[chapter]{quadro}{loq}{\quadroname}
\newlistof{listofquadros}{loq}{\listofquadrosname}
\newlistentry{quadro}{loq}{0}

% configurações para atender às regras da ABNT
\setfloatadjustment{quadro}{\centering}
\counterwithout{quadro}{chapter}
\renewcommand{\cftquadroname}{\quadroname\space} 
\renewcommand*{\cftquadroaftersnum}{\hfill--\hfill}

\setfloatlocations{quadro}{hbtp} % Ver https://github.com/abntex/abntex2/issues/176
% ---

% --- 
% Espaçamentos entre linhas e parágrafos 
% --- 

% O tamanho do parágrafo é dado por:
\setlength{\parindent}{1.3cm}

% Controle do espaçamento entre um parágrafo e outro:
\setlength{\parskip}{0.2cm}  % tente também \onelineskip

% ---
% compila o indice
% ---
\makeindex
% ---

% ----
% Início do documento
% ----
\begin{document}

% Seleciona o idioma do documento (conforme pacotes do babel)
%\selectlanguage{english}
\selectlanguage{brazil}

% Retira espaço extra obsoleto entre as frases.
\frenchspacing 

% ----------------------------------------------------------
% ELEMENTOS PRÉ-TEXTUAIS
% ----------------------------------------------------------
% \pretextual

% ---
% Capa
% ---
\imprimircapa
% ---

% ---
% Folha de rosto
% (o * indica que haverá a ficha bibliográfica)
% ---
\imprimirfolhaderosto*
% ---

% ---
% Inserir a ficha bibliografica
% ---

% Isto é um exemplo de Ficha Catalográfica, ou ``Dados internacionais de
% catalogação-na-publicação''. Você pode utilizar este modelo como referência. 
% Porém, provavelmente a biblioteca da sua universidade lhe fornecerá um PDF
% com a ficha catalográfica definitiva após a defesa do trabalho. Quando estiver
% com o documento, salve-o como PDF no diretório do seu projeto e substitua todo
% o conteúdo de implementação deste arquivo pelo comando abaixo:
%
% \begin{fichacatalografica}
%     \includepdf{fig_ficha_catalografica.pdf}
% \end{fichacatalografica}

\begin{fichacatalografica}
	\sffamily
	\vspace*{\fill}					% Posição vertical
	\begin{center}					% Minipage Centralizado
	\fbox{\begin{minipage}[c][8cm]{13.5cm}		% Largura
	\small
	\imprimirautor
	%Sobrenome, Nome do autor
	
	\hspace{0.5cm} \imprimirtitulo  / \imprimirautor. --
	\imprimirlocal, \imprimirdata-
	
	\hspace{0.5cm} \thelastpage p. : il. (algumas color.) ; 30 cm.\\
	
	\hspace{0.5cm} \imprimirorientadorRotulo~\imprimirorientador\\
	
	\hspace{0.5cm}
	\parbox[t]{\textwidth}{\imprimirtipotrabalho~--~\imprimirinstituicao,
	\imprimirdata.}\\
	
	\hspace{0.5cm}
		1. Algoritmos Evolutivos.
		2. Algoritmos Genéticos.
		3. Otimização.
		I. Orientador.
		II. Universidade de São Paulo - USP.
		III. Instituto de Matemática e Estatística - IME.
		IV. Titulo			
	\end{minipage}}
	\end{center}
\end{fichacatalografica}
% ---

% ---
% Inserir errata
% ---
%\begin{errata}
%Elemento opcional da \citeonline[4.2.1.2]{NBR14724:2011}. Exemplo:
%
%\vspace{\onelineskip}
%
%FERRIGNO, C. R. A. \textbf{Tratamento de neoplasias ósseas apendiculares com
%reimplantação de enxerto ósseo autólogo autoclavado associado ao plasma
%rico em plaquetas}: estudo crítico na cirurgia de preservação de membro em
%cães. 2011. 128 f. Tese (Livre-Docência) - Faculdade de Medicina Veterinária e
%Zootecnia, Universidade de São Paulo, São Paulo, 2011.
%
%\begin{table}[htb]
%\centering
%\footnotesize
%\begin{tabular}{|p{1.4cm}|p{1cm}|p{3cm}|p{3cm}|}
%  \hline
%   \textbf{Folha} & \textbf{Linha}  & \textbf{Onde se lê}  & \textbf{Leia-se}  \\
%    \hline
%    1 & 10 & auto-conclavo & autoconclavo\\
%   \hline
%\end{tabular}
%\end{table}
%
%\end{errata}
% ---

% ---
% Inserir folha de aprovação
% ---

% Isto é um exemplo de Folha de aprovação, elemento obrigatório da NBR
% 14724/2011 (seção 4.2.1.3). Você pode utilizar este modelo até a aprovação
% do trabalho. Após isso, substitua todo o conteúdo deste arquivo por uma
% imagem da página assinada pela banca com o comando abaixo:
%
% \begin{folhadeaprovacao}
% \includepdf{folhadeaprovacao_final.pdf}
% \end{folhadeaprovacao}
%
\begin{folhadeaprovacao}

  \begin{center}
    {\ABNTEXchapterfont\large\imprimirautor}

    \vspace*{\fill}\vspace*{\fill}
    \begin{center}
      \ABNTEXchapterfont\bfseries\Large\imprimirtitulo
    \end{center}
    \vspace*{\fill}
    
    \hspace{.45\textwidth}
    \begin{minipage}{.5\textwidth}
        \imprimirpreambulo
    \end{minipage}%
    \vspace*{\fill}
   \end{center}
        
   Trabalho aprovado. \imprimirlocal, 24 de novembro de 2012:

   \assinatura{\textbf{\imprimirorientador} \\ Orientador} 
   \assinatura{\textbf{Professor} \\ Convidado 1}
   \assinatura{\textbf{Professor} \\ Convidado 2}
   %\assinatura{\textbf{Professor} \\ Convidado 3}
   %\assinatura{\textbf{Professor} \\ Convidado 4}
      
   \begin{center}
    \vspace*{0.5cm}
    {\large\imprimirlocal}
    \par
    {\large\imprimirdata}
    \vspace*{1cm}
  \end{center}
  
\end{folhadeaprovacao}
% ---

% ---
% Dedicatória
% ---
%\begin{dedicatoria}
%   \vspace*{\fill}
%   \centering
%   \noindent
%   \textit{ Este trabalho é dedicado às crianças adultas que,\\
%   quando pequenas, sonharam em se tornar cientistas.} \vspace*{\fill}
%\end{dedicatoria}
% ---

% ---
% Agradecimentos
% ---
%\begin{agradecimentos}
%Os agradecimentos principais são direcionados à Gerald Weber, Miguel Frasson,
%Leslie H. Watter, Bruno Parente Lima, Flávio de Vasconcellos Corrêa, Otavio Real
%Salvador, Renato Machnievscz\footnote{Os nomes dos integrantes do primeiro
%projeto abn\TeX\ foram extraídos de
%\url{http://codigolivre.org.br/projects/abntex/}} e todos aqueles que
%contribuíram para que a produção de trabalhos acadêmicos conforme
%as normas ABNT com \LaTeX\ fosse possível.
%
%Agradecimentos especiais são direcionados ao Centro de Pesquisa em Arquitetura
%da Informação\footnote{\url{http://www.cpai.unb.br/}} da Universidade de
%Brasília (CPAI), ao grupo de usuários
%\emph{latex-br}\footnote{\url{http://groups.google.com/group/latex-br}} e aos
%novos voluntários do grupo
%\emph{\abnTeX}\footnote{\url{http://groups.google.com/group/abntex2} e
%\url{http://www.abntex.net.br/}}~que contribuíram e que ainda
%contribuirão para a evolução do \abnTeX.
%
%\end{agradecimentos}
% ---

% ---
% Epígrafe
% ---
%\begin{epigrafe}
%    \vspace*{\fill}
%	\begin{flushright}
%		\textit{``Não vos amoldeis às estruturas deste mundo, \\
%		mas transformai-vos pela renovação da mente, \\
%		a fim de distinguir qual é a vontade de Deus: \\
%		o que é bom, o que Lhe é agradável, o que é perfeito.\\
%		(Bíblia Sagrada, Romanos 12, 2)}
%	\end{flushright}
%\end{epigrafe}
% ---

% ---
% RESUMOS
% ---

% resumo em português
\setlength{\absparsep}{18pt} % ajusta o espaçamento dos parágrafos do resumo
\begin{resumo}
Um dos problemas mais comuns em matemática é a busca de soluções que maximizem ou minimizem certa função e temos a disposição diversas formas de resolver que podem ser classificadas em técnicas baseadas em cálculo, aleatórias ou métodos que verificam todo o espaço solução pela melhor delas. Dentre as técnicas aleatórias temos aquelas puramente aleatória como o \textit{random walk} e aquelas consideradas aleatórias guiadas como por exemplo o \textit{simulated annealing} ou resfriamento simulado. No âmbito de sistemas evolutivos, os algoritmos genéticos podem ser classificado também como
um método aleatório guiado, onde o estado corrente influencia a próxima escolha para a solução. 

O objetivo desse trabalho é verificar e avaliar os algoritmos genéticos, explorando as técnicas e os mecanismos usados nesse método. Para tal será desenvolvido o algoritmo em uma linguagem de programação usando o paradigma de orientação a objetos e aplicá-lo em um problema de busca pela solução ótima, sendo que buscaremos a melhor solução para um modelo de Ising.
Com isso poderemos avaliar o funcionamento e perfomance do método e como os parâmetros utilizados podem influenciar o resultado

%\textit{Talvez acrescentar que será comparado com outro método como o resfriamento simulado}
	
% Segundo a \citeonline[3.1-3.2]{NBR6028:2003}, o resumo deve ressaltar o
% objetivo, o método, os resultados e as conclusões do documento. A ordem e a extensão
% destes itens dependem do tipo de resumo (informativo ou indicativo) e do
% tratamento que cada item recebe no documento original. O resumo deve ser
% precedido da referência do documento, com exceção do resumo inserido no
% próprio documento. (\ldots) As palavras-chave devem figurar logo abaixo do
% resumo, antecedidas da expressão Palavras-chave:, separadas entre si por
% ponto e finalizadas também por ponto.

 \textbf{Palavras-chave}: Algoritmos Evolutivos. Algoritmos Genéticos. Otimização. Programação Orientada a Objetos.
\end{resumo}

% resumo em inglês
\begin{resumo}[Abstract]
 \begin{otherlanguage*}{english}
   This is the english abstract.

   \vspace{\onelineskip}
 
   \noindent 
   \textbf{Keywords}: Evolutionary Algorithms. Genetic Algorithms. Optimization. Object Oriented Programming.
 \end{otherlanguage*}
\end{resumo}


% ---
% inserir lista de ilustrações
% ---
\pdfbookmark[0]{\listfigurename}{lof}
\listoffigures*
\cleardoublepage
% ---

% ---
% inserir lista de quadros
% ---
\pdfbookmark[0]{\listofquadrosname}{loq}
\listofquadros*
\cleardoublepage
% ---

% ---
% inserir lista de tabelas
% ---
\pdfbookmark[0]{\listtablename}{lot}
\listoftables*
\cleardoublepage
% ---

% ---
% inserir lista de abreviaturas e siglas
% ---
\begin{siglas}
	\item[SGA] Simple Genetic Algorithm
	\item[GA] Genetic Algorithm
	\item[NFL] No-Free-Lunch Theorem
	\item[EA] Evolutionary Algorithm
	\item[OO] Orientação a objetos
\end{siglas}
% ---

% ---
% inserir lista de símbolos
% ---
\begin{simbolos}
  \item[$ \Gamma $] Letra grega Gama
  \item[$ \Lambda $] Lambda
  \item[$ \zeta $] Letra grega minúscula zeta
  \item[$ \in $] Pertence
\end{simbolos}
% ---

% ---
% inserir o sumario
% ---
\pdfbookmark[0]{\contentsname}{toc}
\tableofcontents*
\cleardoublepage
% ---



% ----------------------------------------------------------
% ELEMENTOS TEXTUAIS
% ----------------------------------------------------------
\textual

% ----------------------------------------------------------
% Introdução (exemplo de capítulo sem numeração, mas presente no Sumário)
% ----------------------------------------------------------
% ----------------------------------------------------------
% Arquivo contendo a introdução do trabalho sobre algoritmo genético
%
% ----------------------------------------------------------
\chapter{Introdução}
% ----------------------------------------------------------

\section{Algoritmos evolucionários}

Os algoritmos evolucionários são um subconjunto da computação evolutiva, sendo essa um conjunto de algoritmos para busca de soluções ótimas globais. A computação evolutiva é baseada na evolução das espécies definida na biologia, e assim os algoritmos evolucionários se baseiam em processos encontrados na natureza de seleção, reprodução e mutação das espécies, com o propósito de otimizar a solução para um problema definido.

Um problema de otimização é definido como maximizar, ou minimizar, $f(x)$ sujeito a \(g_i(x) \leq 0\,i = \{1,\ldots, m\}\), e \(h_j(x) = 0, j=\{1,\ldots,n\}\) com \(x \in \Omega \). A solução maximiza, ou minimiza, o escalar \(f(x)\) onde \(x\) é um vetor com dimensão \(n\), \(x=\{x_1,x_2,\ldots,x_n\}\) do espaço de soluções \(\Omega\). \cite{Coello2007} 

Dessa forma o problema de otimização tem os seguintes componentes:
\begin{itemize}
	\item Função objetivo \(f(x)\): função de avaliação que deve se minimizada ou maximizada;
	\item As restrições \(g_i(x)\) e \(h_j(x)\): definem limites para as soluções que são permitidas;
	\item Espaço de soluções \(\Omega\) :
		Conjunto com todas as possíveis soluções para o problema; 
\end{itemize}

De forma mais simplificada podemos escrever o problema de otimização como, sendo \(\Omega\) o espaço de soluções e a função \[ g: \Omega \to \mathbbm{R}\] e a solução o vetor \(x \in \Omega\) tal que \[ \arg \min \limits_{x \in \Omega} g(x)\] que pode facilmente ser convertido para um problema de maximização usando \(-g(x)\)

Para atingir esse objetivo os algoritmos evolutivos trabalham com uma população de indivíduos que indicaremos como soluções candidatas para o problema. Baseado na avaliação de cada individuo com relação ao seu ambiente, em nosso contexto a função de avaliação, e usando operadores definidos para seleção e evolução, a população vai sendo modificada até que se atinja um resultado satisfatório para o problema. Assim como no processo de evolução das espécies definidos por Darwin, que introduziu o conceito de seleção natural  através da sobrevivência do mais apto, os indivíduos da população que tem melhores avaliações, ou seja, que melhor se adaptam ao ambiente, possuem mais probabilidade de sobrevivência e assim se reproduzirem. 

Existem vários métodos para se otimizar um problema conforme pode ser visto na \autoref{fig:classificacao_metodos_busca}. Existem os métodos enumerativos, que se tornam impraticáveis quando o \(\Omega\) é muito grande, pois o algoritmo deveria testar todas as soluções possíveis. Os algoritmos determinísticos são os mais eficientes em determinadas condições, como por exemplo o comportamento de \(f(x)\). Se a função objetivo possui múltiplos máximos (ou mínimos) alguns algoritmos determinísticos, como o \textit{hill-climbing} por exemplo, podem ficar 'presos' em soluções locais e não encontrar a solução global. A última categoria, onde se encontram também os algoritmos evolucionários, são os estocásticos (ou randômicos). Possuem a vantagem de não ficar presos em soluções locais, porém nem sempre obterão a melhor solução.

\begin{figure}
	\includegraphics[width=\linewidth]{imagens/classificacao_metodos_busca.png}
	\caption{Técnicas de otimização global - \cite{Coello2007}}
	\label{fig:classificacao_metodos_busca}
\end{figure}

Dentre a categoria dos métodos estocásticos existem os que são completamente aleatórios, como a busca randômica (ou passeio aleatório), e os que são de alguma forma guiado como o método de Monte Carlo por exemplo. O algoritmo evolucionário se encaixa nessa ultima definição, onde existem componentes aleatórios atuando na seleção e reprodução dos indivíduos mas de forma guiada pelos resultados da função de avaliação.

De acordo com \citeauthor{Linden2008}, os AE são \textbf{heurísticas}\footnote{Heurísticas são algoritmos polinomiais que usualmente tendem a encontrar soluções ótimas ou próximas delas, mas sem garantias} que não asseguram obter o melhor resultado possível, e além disso o resultado pode diferir entre as execuções do algoritmo.

Para \citeauthor{Sivanandam2007}, um algoritmo evolucionário são processos estocásticos e iterativos que operam em um conjunto de indivíduos (população). Cada individuo representa uma possível solução para o problema de otimização ou busca, sendo que os parâmetros estão de alguma forma codificados nesses indivíduos. A população inicial é gerada aleatoriamente e são avaliados usando alguma função, que determina o quão bem o indivíduo responde ao problema. Esse valor determina a direção de busca do algoritmo. 

\section{Biologia}
A ideia por trás dos algoritmos evolucionários e por consequência do algoritmo genético, é a teoria de evolução das espécies na natureza de \textbf{Darwin}, onde a sobrevivência de cada indivíduo é determinada por como ele se adapta ao seu meio. Assim aqueles que conseguiam vantagens sobre os demais por ter uma maior sobrevivência se reproduziam mais, e assim, passavam para as próximas gerações essas características que os diferenciavam. Darwin chamou de seleção natural esse mecanismo da sobrevivência dos mais aptos.

Contudo Darwin não sabia explicar como essa informação era passada dos ancestrais para os descendente, onde então entram as descobertas de \textbf{Mendel}, que através dos conceitos de genética determinava como características eram compartilhadas entre pais e filhos. 

A unidade básica de informação é o gene, que é um bloco de sequências de DNA e o conjunto de genes formam o cromossomo. Cada gene tem um \textit{locus}, que define a região dentro do cromossomo onde está localizado, e possui um conjunto de valores possíveis chamados de alelos. Os genes controlam as características do indivíduo, sendo que a expressão dessas no individuo é denominada de fenótipo. Assim um conjunto específicos de genes define o genótipo do indivíduo que está associado a um fenótipo, que apresenta as características codificadas no genótipo e que podem ser modificadas pelo ambiente. 

Organismos com cromossomos combinados em pares são chamados de diplóides em contraste com os que não possuem pares, chamados de haplóides. Na natureza a maioria dos seres mais complexos que se reproduzem sexualmente são normalmente diplóides e possuem um ou mais pares de cromossomos sendo que sua quantidade e tamanho dependem de cada ser vivo.

Durante a reprodução ocorre a transmissão da informação que pode ser de dois tipos: assexuada e sexuada. Na reprodução assexuada, o organismo replica a si mesmo, presente mais em seres simples, não representa tanta diversidade pois como não existe combinação de material genético entre dois seres, ocorre apenas a retransmissão de material genético, somente sujeito a alterações por mutação na cópia.

Na reprodução sexuada de seres diplóides, há presença de dois indivíduos que compartilham seu material genético para formar um novo organismo. No inicio da reprodução, existe a cópia do material genético e recombinação, também chamado de \textit{crossover}(\autoref{fig:crossover_example}). Esse processo é feito com é feito com os cromossomos se cruzando, por isso o termo \textit{crossover},  um sobre o outro em um ou mais pontos havendo assim a troca nas sequências de genes. Após feita a recombinação, o material genético é divido em gametas que então são combinados com os gametas do outro pai para formar novamente um cromossomo diplódie completo, gerando assim os novos indivíduos. Para organismos haplóides é feita apenas a combinação das sequencias de cada pai.

\begin{figure}
	\begin{center}
	\includegraphics[width=0.5\textwidth]{imagens/cross_over.png}
	\caption{Exemplo de reprodução com crossover - \cite{Klug2011}}
	\label{fig:crossover_example}
	\end{center}
\end{figure}


Dentro da etapa de replicação do DNA podem ocorrer erros ou alterações influenciadas por fatores externos, gerando assim as mutações. Isso pode ser positivo, negativo ou não influenciar o resultado final, mas é um outro mecanismo que pode determinar a evolução das espécies.   

Mendel ainda definiu o conceito de dominância-recessividade em organismos diplóides, assumindo que cada característica é controlada por pares de genes, sendo recebidos um de cada pai. Assim um dos alelos do gene é dominante sobre o outro apresentando a característica final no individuo, e outro alelo será considerado recessivo.

Combinando indivíduos que melhor se adaptaram as condições de sobrevivência, provavelmente deve gerar novos indivíduos que tenha características ainda melhores que seus antecessores. O outro caso pode acontecer também e o novo indivíduo ter piores condições de viver, e assim o processo de seleção natural levará esse espécime a extinção favorecendo outros que se sobressaíram na próxima geração.

A evolução então é um processo adaptativo onde através de mutações e recombinação entre os indivíduos, vão surgindo novas gerações que devem apresentar cada vez mais seres que se adaptam cada vez melhor ao ambiente.

\section{Algoritmo genético}

Os algoritmos evolucionários vem sendo estudados a um longo tempo, desde da década de 40 que os cientistas se inspiram na natureza para criar os primeiros passos para a inteligência artificial, passando pela década de 50 onde começam os estudos sobre sistemas adaptativos para gerar soluções candidatas para problemas de difícil solução. Na década de 60 Rechenberg desenvolveu as estratégias evolucionárias (\textit{evolutionary strategies}) usando cromossomos compostos de números reais para estudos com aerofólios. Também apareceu a programação evolucionária, usando estruturas de pequenas máquinas de estados para resolver determinadas tarefas, que evolui para programação genética onde pequenos programas passam a ser as soluções candidatas. Existiram outros trabalhos sobre algoritmos evolucionários, programação evolucionária e algoritmos genéticos nas áreas de otimização e aprendizado de máquina, porém foram os trabalhos de Holland nas décadas de 60 e 70 que consolidou os algoritmos genéticos. \cite{Mitchell1996, Linden2008}

Alguns autores como \citeauthor{Mitchell1996}, \citeauthor{LeeJacobson2015}, \citeauthor{Kwong2001} entre outros citam Holland como criador do algoritmo genético(\textit{Genethic Algorithm} ou GA) em 1975 no livro \textit{"Adaptation in Natural and Artificial Systems"}. Nele Holland formaliza uma estrutura para sistemas adaptativos, e encaixa o GA como uma abstração para a evolução biológica.

Na estrutura criada por Holland para um problema bem posto de adaptação temos:
\begin{itemize}
	\item \( \mathscr{A} = \{ \mathcal{A}_1, \mathcal{A}_2, \ldots \} \) é um conjunto de estruturas do domínio.
	\item \( \Omega = \{ \omega_1, \omega_2, \ldots \} \) é um conjunto de operadores para modificar as estruturas de \( \mathscr{A} \).  Sendo \( \omega \in \Omega \) uma função \(\omega : \mathscr{A} \to \mathcal{P} \) onde \(\mathcal{P}\) é uma distribuição de probabilidades sobre \(\mathscr{A}\)
	\item \( \mathcal{I}\) é o conjunto de entradas do ambiente
	\item \( \tau : \mathcal{I} \times \mathscr{A} \to \Omega\) é o plano adaptativo baseado nas entradas e nas estruturas no intervalo de tempo \(t\) e determina os operadores a serem aplicados nesse intervalo.
\end{itemize}
Assim: 
\[\tau(\mathcal{I}(t), \mathscr{A}(t)) = \omega_t \in \Omega \text{ e } \omega(t)(\mathscr{A}(t)) = \mathcal{P}(t + 1)\]
onde \(\mathcal{P}(t + 1) \) é uma distribuição particular sobre \( \mathscr{A} \). \( \mathscr{A}(t+1) \) é determinado retirando uma amostra aleatória de \( \mathscr{A} \) de acordo com essa distribuição. \cite{Holland1992}

Ainda define os seguintes itens, \( \mathcal{T} \) como o conjunto de planos possíveis (\(\tau\)), \(\mathcal{E}\) como sendo o conjunto de ambientes possíveis e dado \( E \in \mathcal{E} \) a função de avaliação \(\mu_E(\mathscr{A}(t))\). Por fim define o \(\mathcal{X})\) como critério de comparação entre os planos em \( \mathcal{T} \) sobre \(\mathcal{E}\).

Usando essa estrutura com a define o algoritmo genético na forma:
\begin{itemize}
	\item \( \mathscr{A} \), como população de cromossomos
	\item \( \Omega \), operadores genéticos de inversão, mutação, crossover, dominância, etc.
	\item \( \mathcal{T} \), planos de reprodução baseados na avaliação em conjunto com os operadores genéticos.
\end{itemize}


% ----------------------------------------------------------

% ----------------------------------------------------------
% PARTE
% ----------------------------------------------------------
%\part{Preparação da pesquisa}
% ----------------------------------------------------------

% ---
% Capitulo com exemplos de comandos inseridos de arquivo externo 
% ---
%\include{abntex2-modelo-include-comandos}
% ---

% ---
% Capitulo explicando teoria sobre o algoritmo genético
% ---
% ---
% Este capítulo, apresenta os conceitos sobre o algoritmo genético
% ---

\chapter{O Algoritmo Genético}

\section{Breve história}

Lorem ipsum. \autoref{eqn:teste}

\begin{equation}
\label{eqn:teste}
x = y
\end{equation}

% ---
% Capitulo explicando modelo de Ising
% ---
%\include{isingModel}

% ---
% Capitulo explicando implementacao
% ---
% ---
% Este capítulo, apresenta os conceitos sobre o algoritmo genético
% ---

\chapter{Implementação do algoritmo genético}
\label{chap:implementGA}

Para implementação do algoritmo foi utilizado a linguagem de programação JAVA de forma a aproveitar o paradigma de orientação a objetos com a característica de reaproveitamento de código. A estrutura do programa permite com classes básicas formar a lógica do algoritmo e com classes derivadas especializar o algoritmo para especificações de cada problema. Dessa forma com pouco código extra sendo feito é possível reaproveitar as classes já definidas, e usando o polimorfismo inerente a linguagens OO, é possível reescrever e acrescentar as partes necessárias para embutir conhecimento específico no problema.

Os testes começam sobre um modelo simples usando codificação binária para representar duas variáveis reais \textit{x} e \textit{y} com valores no intervalo \([-100,100]\), e 15 bits de comprimento. A conversão é feita usando um fator \textit{C} definido por \(C = \dfrac{200}{2^15}\) que multiplicado pela representação binária transformada para o inteiro \textit{i} e somado ao valor mínimo, apresenta o valor real da variável, com resolução definida pelo fator.
\(x = C * i_x - 100\). O cromossomo é formado por uma sequência de 30 bits, 15 para cada variável, concatenados na forma de um vetor (De fato em Java será utilizado os conceitos de Listas por serem mais versáteis com relação aos elementos que podem ser guardados e manipulados). A função que deve ser maximizada é dada por: 
\[f = \left| x \cdot y \cdot \sin\left(\frac{\pi \cdot y}{4}\right) \right|\]
Essa mesma função será usada para a avaliação e apresenta o seguinte gráfico:


\section{Modelo de Ising}

O modelo de Ising\footnote{O modelo também é conhecido como Lenz-Ising, pois Lenz introduziu o modelo, porém nunca calculou nenhuma de suas propriedades} é um modelo da física para estudos de fênomenos magnéticos em materiais. Ising calculou o modelo para uma dimensão e depois Onsanger calculou para uma rede quadrada (2D). Apesar do modelo ter sido elaborado para compreender melhor o propriedades magnéticas de certos materiais, ele se mostrou de grande utilidade para modelar problemas de outras áreas de estudos de fenômenos cooperativos, como por exemplo a dissertação de \cite{Lucena2014} onde usa o modelo de Ising aplicado a um estudo de criminalidade. É considerado um dos mais simples e mais estudado dos modelos de mecânica estatística.

O modelo define uma grade de \textit{spins} \(\sigma_i\) que poderiam conter os valores -1 e +1 somente. Esses spins possuem uma interação de energia dada por \(-E(i, i')\sigma_i\sigma_{i'}\), ou seja, o valor de energia de dois spins será \(-E(i, i')\) se os spins forem diferentes, e \(+E(i, i')\) se forem iguais, e adicionalmente pode interagir com um campo magnético externo \textit{H} com energia \(-H\sigma_i\). \cite{McCoy1973}.

O modelo mais simples usa uma grade quadrangular, facilmente representada por uma matriz, e com \(E(i, i')\) um valor fixo independente dos spins e que se \textit{i} e \textit{i'} não forem vizinhos será igual 0.



O algoritmo genético simples será usado para apresentar uma solução para modelo de Ising em uma matriz \(10 \times 10\), com cada nó (spin) representado pelos possíveis valores $-1$ e $1$. 

Sendo \(\Lambda = \left[-5,5\right]^2\) as posições da matriz, e \(\sigma_t \in \{-1,1\}\) representando os valores de cada spin em determinada posição da matriz, e seja \( (\sigma_t)_{t \in \Lambda} = \sigma \) um conjunto de valores da matriz. 

A função de avaliação que deve ser minimizada é dada por \( H(\sigma) = -\sum \limits_{<t,t'>}\sigma_t \sigma_{t'} \) onde $t$ e $t'$ representam dois spins vizinhos.   

\section{Codificação}

A classe cromossomo será usada para representar cada solução para o problema e conterá um vetor de valores inteiros de tamanho $100$, representando assim a matriz \( 10 \times 10 \), alinhando uma linha da matriz na sequencia da outra no vetor. Assim o nosso genótipo é uma cadeia de inteiros, onde o fenótipo será a matriz correspondente quebrando o vetor em linhas de tamanho $10$.

Exemplo da relação genótipo x fenótipo para uma matriz \( 3 \times 3 \):
\begin{align*}
\text{cromossomo} &= \left[-1, 1, 1, 1, -1, 1, 1, 1, -1 \right]\\
\text{fenótipo} &= \begin{bmatrix}
-1 & 1 & 1 \\
1 & -1 & 1 \\
1 & 1 & -1
\end{bmatrix}
\end{align*}


\section{Seleção e avaliação}

\section{Crossover e mutação}

\section{Resultados}


%\chapter{Conteúdos específicos do modelo de trabalho acadêmico}\label{cap_trabalho_academico}
%
%\section{Quadros}
%
%Este modelo vem com o ambiente \texttt{quadro} e impressão de Lista de quadros 
%configurados por padrão. Verifique um exemplo de utilização:
%
%\begin{quadro}[htb]
%\caption{\label{quadro_exemplo}Exemplo de quadro}
%\begin{tabular}{|c|c|c|c|}
%	\hline
%	\textbf{Pessoa} & \textbf{Idade} & \textbf{Peso} & \textbf{Altura} \\ \hline
%	Marcos & 26    & 68   & 178    \\ \hline
%	Ivone  & 22    & 57   & 162    \\ \hline
%	...    & ...   & ...  & ...    \\ \hline
%	Sueli  & 40    & 65   & 153    \\ \hline
%\end{tabular}
%\fonte{Autor.}
%\end{quadro}
%
%Este parágrafo apresenta como referenciar o quadro no texto, requisito
%obrigatório da ABNT. 
%Primeira opção, utilizando \texttt{autoref}: Ver o \autoref{quadro_exemplo}. 
%Segunda opção, utilizando  \texttt{ref}: Ver o Quadro \ref{quadro_exemplo}.

%% ----------------------------------------------------------
%% PARTE
%% ----------------------------------------------------------
%\part{Referenciais teóricos}
%% ----------------------------------------------------------
%
%% ---
%% Capitulo de revisão de literatura
%% ---
%\chapter{Lorem ipsum dolor sit amet}
%% ---
%
%% ---
%\section{Aliquam vestibulum fringilla lorem}
%% ---
%
%\lipsum[1]
%
%\lipsum[2-3]

% ----------------------------------------------------------
% PARTE
% ----------------------------------------------------------
%\part{Resultados}
% ----------------------------------------------------------

%% ---
%% primeiro capitulo de Resultados
%% ---
%\chapter{Lectus lobortis condimentum}
%% ---
%
%% ---
%\section{Vestibulum ante ipsum primis in faucibus orci luctus et ultrices
%posuere cubilia Curae}
%% ---
%
%\lipsum[21-22]
%
%% ---
%% segundo capitulo de Resultados
%% ---
%\chapter{Nam sed tellus sit amet lectus urna ullamcorper tristique interdum
%elementum}
%% ---
%
%% ---
%\section{Pellentesque sit amet pede ac sem eleifend consectetuer}
%% ---
%
%\lipsum[24]

% ----------------------------------------------------------
% Finaliza a parte no bookmark do PDF
% para que se inicie o bookmark na raiz
% e adiciona espaço de parte no Sumário
% ----------------------------------------------------------
\phantompart

% ---
% Conclusão
% ---
\chapter{Conclusão}
% ---
%\lipsum[31-33]
O algoritmo genético é uma outra ferramenta que dispomos pela busca de soluções ótimas. Como visto na literatura, podemos considerar que na verdade é um algoritmo com foco em busca de soluções satisfatórias do que exatamente da solução ótima. Uma das vantagens está na facilidade de implementação do algoritmo, porém possui parâmetros de configuração que ainda não se conhece formas de definir analiticamente os melhores valores, além de dependerem das condições do problema a ser resolvido. Com isso a maior parte dos trabalhos se baseiam em resultados empíricos para ajuste dos parâmetros. Foi visto que o GA também apresenta algumas dificuldades com certas condições, mas que podem ser amenizadas se propriamente codificado e parametrizado. 

Nos testes realizados, o GA se mostrou bem mais eficiente que a busca aleatória, alcançando resultados mais expressivos. Portanto em problemas que por limitações inerentes seriam usados o algoritmo de busca aleatória, o algoritmo genético se apresenta como uma outra abordagem possível. 

Ficou demonstrado que é muito importante as escolhas feitas para a arquitetura do GA e seus parâmetros podem definir o sucesso do algoritmo. No teste sobre o modelo de Ising, percebeu-se que a utilização de uma função de avaliação, que não era a mais apropriada, resultou em ineficiência do algoritmo obtendo resultados iguais e até inferiores que a busca aleatória, fato contornado usando outra função. Esse tipo de detalhe é exatamente o \citeauthor{Linden2008} menciona sobre quanto mais informação sobre o problema conseguir inserir no algoritmo, melhor será a resposta, e por isso não existe uma padrão do GA que seja ideal para todos os problemas.

Outro conceito que impôs diferença nos resultados foi o uso do elitismo, pois ao manter os melhores cromossomos de uma geração para outra, sustentou-se o melhor resultado até o fim do processo. Sem o elitismo era possível ver nos gráficos que o melhor cromossomo em gerações \(t+1\) ficavam com pior resultado que o o melhor da geração \(t\).

No modelo de Ising, o algoritmo mostrou através da análise das médias e variâncias, que caminhava para uma convergência, e com boa probabilidade de se tiver gerações suficientes, encontrar o estado de menor energia

As aplicações para o GA são as mais variadas, podendo ser usado em problemas de busca de máximo ou mínimo de funções multivariadas e também com multiobjetivos, na busca de ordenações mais eficientes, como é o caso do problema do caixeiro viajante, e programação genética, onde o GA consegue criar pequenos algoritmos para solução de problemas. 

Umas das aplicações interessantes para o algoritmo é na busca dos coeficientes de uma rede neural. Para as redes usadas em processos de \textit{machine learning} para classificação, pode ser usado de forma a determinar os coeficientes iniciais para o algoritmo de \textit{backpropagation}, que por ser uma função com vários máximos locais, dependendo do ponto inicial utilizado pode ficar preso a um que não seja o máximo global. Normalmente o ponto inicial é definido aleatoriamente, e nesse sentido que poderia ser usado o GA, que teria uma população definida por cromossomos de valores reais definindo os coeficientes inicias da rede. Para cada cromossomo poderia ser executado o \textit{backpropagation} e verificado através dos resultados qual obteve melhor taxa de sucesso de classificação, e usado esse valor como função de avaliação.

Ainda com redes neurais, existem os casos que a rede define o comportamento de um sistema em resposta a sensores de entrada. O caso clássico de demonstração dessa aplicação é do treinar o computador para jogar determinado jogo eletrônico. A população fica definida como cromossomos de valores reais com os coeficientes da rede e o resultado da rede só pode ser visto após determinada simulação. A população do GA testada contra o resultado de pontuação, que será a função de avaliação do algoritmo, é evoluída então combinando os melhores resultados e aplicando mutação em alguns.

Outra característica interessante, que ainda pode ser explorada, é a possibilidade de combinar o GA com outro algoritmo de otimização, por exemplo direcionando a mutação para uma busca de máximo local da função de avaliação. Esse pode ser um diferencial combinando a qualidade do GA de encontrar soluções satisfatórias ou próximas a máximos locais e combinar com um algoritmo mais eficiente de maximização local, como o \textit{hill climbing}.

O algoritmo genético ainda precisa ser muito analisado de forma a entender melhor seu mecanismo de funcionamento e dessa forma conseguir determinar os melhores parâmetros, e tem demonstrado certa popularidade com as novas técnicas de AI e \textit{machine learning}.

Para trabalhos futuros podem ser seguido os testes com o modelo proposto de Ising, trabalhando com contornos e busca de soluções para acoplamentos do modelo, também podem ser explorados as técnicas de combinar com outros métodos. Ainda é necessário também avaliar o algoritmo com relação a seu desempenho computacional e outro passo importante seria o de executar o algoritmo usando paralelismo, pois a fase de avaliação e seleção poderia facilmente ser colocada em tarefas concorrentes, otimizando o tempo de resposta do algoritmo.  


% ----------------------------------------------------------
% ELEMENTOS PÓS-TEXTUAIS
% ----------------------------------------------------------
\postextual
% ----------------------------------------------------------

% ----------------------------------------------------------
% Referências bibliográficas
% ----------------------------------------------------------
%\bibliography{abntex2-modelo-references}
\nocite{LeeJacobson2015} % colocar nessa forma todos aqueles que não foram citados diretamente
\bibliography{TC}

% ----------------------------------------------------------
% Glossário
% ----------------------------------------------------------
%
% Consulte o manual da classe abntex2 para orientações sobre o glossário.
%
%\glossary

% ----------------------------------------------------------
% Apêndices
% ----------------------------------------------------------

% ---
% Inicia os apêndices
% ---
%\begin{apendicesenv}

% Imprime uma página indicando o início dos apêndices
%\partapendices

%% ----------------------------------------------------------
%\chapter{Quisque libero justo}
%% ----------------------------------------------------------
%
%\lipsum[50]
%
%% ----------------------------------------------------------
%\chapter{Nullam elementum urna vel imperdiet sodales elit ipsum pharetra ligula
%ac pretium ante justo a nulla curabitur tristique arcu eu metus}
%% ----------------------------------------------------------
%\lipsum[55-57]

%\end{apendicesenv}
% ---


% ----------------------------------------------------------
% Anexos
% ----------------------------------------------------------

% ---
% Inicia os anexos
% ---
\begin{anexosenv}

% Imprime uma página indicando o início dos anexos
%\partanexos

% ---
% Este capítulo, apresenta os anexos
% ---

\chapter{Resultados}
\label{chap:anexos}

\begin{table}[h!]
	\begin{tabular}{|c|c|c|c|c|}
		\hline
		\textbf{Cromossomo} 	& \textbf{Avaliação} 	& \textbf{\(\mathcal{H}(\sigma) \)}	& \textbf{Taxa de seleção}	& \textbf{Proporção estimada} \\
		\hline
		287	&  222	&  42 	&  0,067	&  0,040 \\ \hline 
		277	&  206	&  26 	&  0,050	&  0,037 \\ \hline 
		280	&  204	&  24 	&  0,033	&  0,037 \\ \hline 
		282	&  204	&  24 	&  0,017	&  0,037 \\ \hline 
		285	&  200	&  20 	&  0,000	&  0,036 \\ \hline 
		276	&  194	&  14 	&  0,000	&  0,035 \\ \hline 
		291	&  194	&  14 	&  0,083	&  0,035 \\ \hline 
		286	&  192	&  12 	&  0,017	&  0,034 \\ \hline 
		293	&  192	&  12 	&  0,033	&  0,034 \\ \hline 
		283	&  190	&  10 	&  0,033	&  0,034 \\ \hline 
		288	&  190	&  10 	&  0,017	&  0,034 \\ \hline 
		289	&  190	&  10 	&  0,000	&  0,034 \\ \hline 
		299	&  188	&  8 	&  0,017	&  0,034 \\ \hline 
		273	&  186	&  6 	&  0,067	&  0,033 \\ \hline 
		300	&  186	&  6 	&  0,017	&  0,033 \\ \hline 
		292	&  184	&  4 	&  0,033	&  0,033 \\ \hline 
		274	&  182	&  2 	&  0,000	&  0,033 \\ \hline 
		297	&  182	&  2 	&  0,067	&  0,033 \\ \hline 
		271	&  180	&  0 	&  0,033	&  0,032 \\ \hline 
		281	&  180	&  0 	&  0,033	&  0,032 \\ \hline 
		295	&  180	&  0 	&  0,050	&  0,032 \\ \hline 
		290	&  178	&  -2 	&  0,050	&  0,032 \\ \hline 
		294	&  178	&  -2 	&  0,033	&  0,032 \\ \hline 
		278	&  176	&  -4 	&  0,017	&  0,032 \\ \hline 
		284	&  176	&  -4 	&  0,067	&  0,032 \\ \hline 
		279	&  174	&  -6 	&  0,033	&  0,031 \\ \hline 
		275	&  172	&  -8 	&  0,067	&  0,031 \\ \hline 
		298	&  172	&  -8 	&  0,033	&  0,031 \\ \hline 
		272	&  164	&  -16 	&  0,000	&  0,029 \\ \hline 
		296	&  160	&  -20 	&  0,033	&  0,029 \\
		\hline
	\end{tabular}
	\caption{Resumo da geração 9 com média de avaliação 185,86 mostrando a taxa de amostragem da seleção e o que era esperado para a proporção usando a função de avaliação da \autoref{eq:Energia_modelo_ising}}
	\label{tab:resumo_GA_H}
\end{table}
\clearpage

\begin{table}[h!]
	\begin{tabular}{|c|c|c|c|c|}
		\hline
		\textbf{Cromossomo} 	& \textbf{Avaliação} 	& \textbf{\(\mathcal{H}(\sigma) \)}	& \textbf{Taxa de seleção}	& \textbf{Proporção estimada} \\
		\hline
		178	&  6,69		&  -38 &  0,100	&  0,063\\ \hline
		160	&  6,05		&  -36 &  0,100	&  0,057\\ \hline
		159	&  5,47		&  -34 &  0,033	&  0,052\\ \hline
		166	&  5,47		&  -34 &  0,067	&  0,052\\ \hline
		162	&  4,95		&  -32 &  0,067	&  0,047\\ \hline
		171	&  4,95		&  -32 &  0,000	&  0,047\\ \hline
		180	&  4,95		&  -32 &  0,033	&  0,047\\ \hline
		167	&  4,48		&  -30 &  0,050	&  0,042\\ \hline
		175	&  4,48		&  -30 &  0,017	&  0,042\\ \hline
		161	&  4,06		&  -28 &  0,050	&  0,038\\ \hline
		163	&  4,06		&  -28 &  0,033	&  0,038\\ \hline
		169	&  4,06		&  -28 &  0,017	&  0,038\\ \hline
		151	&  3,67		&  -26 &  0,050	&  0,035\\ \hline
		155	&  3,67		&  -26 &  0,033	&  0,035\\ \hline
		170	&  3,32		&  -24 &  0,017	&  0,031\\ \hline
		176	&  3,32		&  -24 &  0,050	&  0,031\\ \hline
		153	&  3,00		&  -22 &  0,067	&  0,028\\ \hline
		157	&  3,00		&  -22 &  0,033	&  0,028\\ \hline
		174	&  3,00		&  -22 &  0,050	&  0,028\\ \hline
		152	&  2,72		&  -20 &  0,000	&  0,026\\ \hline
		164	&  2,72		&  -20 &  0,050	&  0,026\\ \hline
		165	&  2,72		&  -20 &  0,000	&  0,026\\ \hline
		158	&  2,46		&  -18 &  0,017	&  0,023\\ \hline
		154	&  2,23		&  -16 &  0,033	&  0,021\\ \hline
		168	&  2,01		&  -14 &  0,000	&  0,019\\ \hline
		173	&  2,01		&  -14 &  0,000	&  0,019\\ \hline
		156	&  1,82		&  -12 &  0,033	&  0,017\\ \hline
		179	&  1,82		&  -12 &  0,000	&  0,017\\ \hline
		172	&  1,22		&  -4 &  0,000	&  0,012\\ \hline
		177	&  1,11		&  -2 &  0,000	&  0,010\\
		\hline
	\end{tabular}
	\caption{Resumo da geração 5 com média de avaliação 3.516 mostrando a taxa de amostragem da seleção e o que era esperado para a proporção usando a função de avaliação da \autoref{eq:selecao_modelo_ising}}
	\label{tab:resumo_GA_expH}
\end{table}

\clearpage
\begin{figure}[h!]
	\centering
	\begin{subfigure}[b]{0.47\linewidth}
		\includegraphics[width=\linewidth]{imagens/graph_pc_0_950_pm_0_000_pop_50_g_300__2.png}
		\caption{}
	\end{subfigure}
	\begin{subfigure}[b]{0.47\linewidth}
		\includegraphics[width=\linewidth]{imagens/graph_pc_0_950_pm_0_000_pop_50_g_300__3.png}
		\caption{}
	\end{subfigure}
	\begin{subfigure}[b]{0.47\linewidth}
		\includegraphics[width=\linewidth]{imagens/graph_pc_0_950_pm_0_001_pop_50_g_300__2.png}
		\caption{}
	\end{subfigure}
	\begin{subfigure}[b]{0.47\linewidth}
		\includegraphics[width=\linewidth]{imagens/graph_pc_0_950_pm_0_001_pop_50_g_300__3.png}
		\caption{}
	\end{subfigure}
	\begin{subfigure}[b]{0.47\linewidth}
		\includegraphics[width=\linewidth]{imagens/graph_pc_0_950_pm_0_020_pop_50_g_300__1.png}
		\caption{}
	\end{subfigure}
	\begin{subfigure}[b]{0.47\linewidth}
		\includegraphics[width=\linewidth]{imagens/graph_pc_0_950_pm_0_020_pop_50_g_300__3.png}
		\caption{}
	\end{subfigure}
	\begin{subfigure}[b]{0.47\linewidth}
		\includegraphics[width=\linewidth]{imagens/graph_pc_0_950_pm_0_020_pop_50_g_300__2_noelite.png}
		\caption{Sem elitismo}
	\end{subfigure}
	\begin{subfigure}[b]{0.47\linewidth}
		\includegraphics[width=\linewidth]{imagens/graph_pc_0_950_pm_0_020_pop_50_g_300__3_noelite.png}
		\caption{Sem elitismo}
	\end{subfigure}
	\caption{Evolução do algoritmo genético durante gerações}
	\label{fig:evolucaoGA2}
\end{figure}
\clearpage



	\begin{lstlisting}[caption={Resultado de um ciclo de seleção, crossover e mutação do GA para o modelo de Ising}, label=qd:resultado_reproducao_ising] 
Geração: 1 | Média Avaliação: 1.601593
#### CHILD #####
ID: 54
Fenotipo: 
1 | -1 |  1 | -1 | -1 |  1 |  1 | -1 | -1 |  1
--------------------------------------------------
1 |  1 |  1 | -1 | -1 |  1 |  1 | -1 | -1 |  1
--------------------------------------------------
1 |  1 | -1 | -1 |  1 |  1 |  1 |  1 | -1 |  1
--------------------------------------------------
1 |  1 | -1 | -1 |  1 | -1 |  1 |  1 |  1 | -1
--------------------------------------------------
1 |  1 | -1 |  1 | -1 | -1 |  1 |  1 |  1 | -1
--------------------------------------------------
1 |  1 | -1 | -1 | -1 | -1 |  1 |  1 |  1 | -1
--------------------------------------------------
1 |  1 | -1 |  1 | -1 | -1 |  1 |  1 |  1 |  1
--------------------------------------------------
1 |  1 |  1 | -1 | -1 |  1 |  1 |  1 |  1 |  1
--------------------------------------------------
-1 |  1 | -1 | -1 |  1 | -1 | -1 | -1 |  1 | -1
--------------------------------------------------
1 | -1 | -1 |  1 |  1 |  1 | -1 |  1 |  1 |  1

#1:  60 | #-1:  40

Avaliação: 6.685894
#### PARENT 1 #####
ID: 11
Fenotipo: 
1 | -1 | -1 | -1 | -1 | -1 |  1 | -1 | -1 |  1
--------------------------------------------------
-1 |  1 |  1 | -1 | -1 |  1 |  1 | -1 | -1 | -1
--------------------------------------------------
1 |  1 | -1 | -1 | -1 |  1 |  1 |  1 | -1 |  1
--------------------------------------------------
1 |  1 |  1 |  1 |  1 |  1 |  1 | -1 |  1 | -1
--------------------------------------------------
-1 |  1 | -1 |  1 | -1 | -1 | -1 |  1 |  1 | -1
--------------------------------------------------
1 |  1 |  1 | -1 | -1 | -1 |  1 |  1 |  1 | -1
--------------------------------------------------
1 | -1 | -1 |  1 |  1 | -1 | -1 |  1 |  1 |  1
--------------------------------------------------
1 | -1 | -1 | -1 | -1 |  1 |  1 | -1 |  1 | -1
--------------------------------------------------
-1 |  1 | -1 | -1 |  1 | -1 | -1 | -1 |  1 | -1
--------------------------------------------------
1 | -1 | -1 |  1 |  1 |  1 |  1 |  1 |  1 |  1

#1:  52 | #-1:  48

Avaliação: 1.000000
Selecionado 2 vezes, 0.04

#### PARENT 2 #####
ID: 13
Fenotipo: 
1 |  1 |  1 | -1 | -1 |  1 |  1 | -1 |  1 |  1
--------------------------------------------------
1 |  1 |  1 |  1 | -1 |  1 | -1 | -1 | -1 |  1
--------------------------------------------------
-1 |  1 |  1 |  1 |  1 |  1 | -1 | -1 | -1 |  1
--------------------------------------------------
-1 | -1 | -1 | -1 |  1 | -1 |  1 | -1 | -1 | -1
--------------------------------------------------
1 | -1 | -1 | -1 | -1 | -1 |  1 | -1 | -1 | -1
--------------------------------------------------
-1 |  1 | -1 |  1 | -1 |  1 |  1 | -1 | -1 | -1
--------------------------------------------------
1 |  1 | -1 | -1 | -1 | -1 |  1 |  1 |  1 | -1
--------------------------------------------------
1 | -1 |  1 |  1 |  1 |  1 |  1 |  1 |  1 |  1
--------------------------------------------------
-1 | -1 | -1 | -1 |  1 | -1 | -1 |  1 |  1 | -1
--------------------------------------------------
1 |  1 | -1 |  1 |  1 |  1 | -1 |  1 |  1 |  1

#1:  52 | #-1:  48

Avaliação: 4.055200
Selecionado 11 vezes, 0.22

### HERITAGE MAP ###
2 |  1 |  2 |  1 |  1 |  2 |  1 |  2 |  1 |  2
--------------------------------------------------
2 |  1 |  1 |  1 |  2 |  1 |  1 |  1 |  2 |  2
--------------------------------------------------
1 |  1 |  1 |  1 |  2 |  2 |  1 |  1 |  2 |  1
--------------------------------------------------
1 |  1 |  2 |  2 |  2 |  2 |  2 |  2 |  1 |  1
--------------------------------------------------
2 |  1 |  1 |  1 |  2 |  2 |  2 |  1 |  1 |  1
--------------------------------------------------
1 |  2 |  2 |  1 |  1 |  1 |  2 |  1 |  1 |  2
--------------------------------------------------
1 |  2 |  2 |  1 |  2 |  2 |  2 |  2 |  1 |  1
--------------------------------------------------
2 |  1 |  2 |  1 |  1 |  2 |  2 |  2 |  2 |  2
--------------------------------------------------
2 |  1 |  2 |  2 |  1 |  1 |  1 |  1 |  1 |  1
--------------------------------------------------
1 |  1 |  2 |  2 |  1 |  2 |  2 |  1 |  2 |  2

### MUTATE MAP
FA | FA | FA | FA | FA | FA | FA | FA | FA | FA
--------------------------------------------------
FA | FA | FA | FA | FA | FA | FA | FA | FA | FA
--------------------------------------------------
FA | FA | FA | FA | FA | FA | FA | FA | FA | FA
--------------------------------------------------
FA | FA | FA | FA | FA | FA | FA | TR | FA | FA
--------------------------------------------------
FA | FA | FA | FA | FA | FA | FA | FA | FA | FA
--------------------------------------------------
FA | FA | FA | FA | FA | FA | FA | FA | FA | FA
--------------------------------------------------
FA | FA | FA | FA | FA | FA | FA | FA | FA | FA
--------------------------------------------------
FA | TR | FA | FA | FA | FA | FA | FA | FA | FA
--------------------------------------------------
FA | FA | FA | FA | FA | FA | FA | FA | FA | FA
--------------------------------------------------
FA | FA | FA | FA | FA | FA | FA | FA | FA | FA
	\end{lstlisting}


%% ---
%\chapter{Morbi ultrices rutrum lorem.}
%% ---
%\lipsum[30]
%
%% ---
%\chapter{Cras non urna sed feugiat cum sociis natoque penatibus et magnis dis
%parturient montes nascetur ridiculus mus}
%% ---
%
%\lipsum[31]
%
%% ---
%\chapter{Fusce facilisis lacinia dui}
%% ---
%
%\lipsum[32]

\end{anexosenv}

%---------------------------------------------------------------------
% INDICE REMISSIVO
%---------------------------------------------------------------------
\phantompart
\printindex
%---------------------------------------------------------------------

\end{document}
